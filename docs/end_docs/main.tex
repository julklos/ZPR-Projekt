\documentclass{article}

% Polskie znaki
\usepackage{polski}
\usepackage[polish]{babel}
\usepackage[utf8]{inputenc}
\usepackage[T1]{fontenc}
\usepackage[utf8]{luainputenc}
\usepackage{indentfirst}

\title{\LARGE{Dokumentacja końcowa projektu} \\ \Large{Serwer serii danych}}
\author{Julia Kłos \and Jakub Sikora}
\date{\today}

\begin{document}
\maketitle

\section{Instrukcja użytkownika}
W celu uruchomienia projektu należy otworzyć przeglądarkę internetową i 
w miejsce adresu strony wpisać \texttt{localhost:3000}. Spowoduje to załadowanie
aplikacji klienckiej. Użytkownik w celu uzyskania dostępu, musi uwierzytelnić się
przy pomocy loginu i hasła, otrzymanego od administratora serwisu, wykorzystując przy 
tym specjalnie przygotowany formularz do logowania. W przypadku błędnych danych
uwierzytelniających, użytkownik nie otrzyma dostępu do aplikacji i zostanie 
poinformowany o błędzie, stosownym komunikatem. Poprawne dane uwierzytelniające, sprawią
że użytkownik zostanie przekierowany na panel operatora procesu.\\


\subsection{Panel operatora}
Panel operatora procesu składa się z logicznie wydzielonych czterech podpaneli, z których
każdy reprezentuje jeden czujnik obsługiwanego procesu. W górnej części podpanelu wyświetlana
jest wartość danej zmiennej procesowej, aktualizowana na bieżąco. Pod wartością, znajdują się 
pozostałe informacje na temat danej zmiennej procesowej, takie jak: 
\begin{itemize}
    \item opis czujnika
    \item model 
    \item identyfikatora czujnika w systemie
\end{itemize}

Panel operatora nie posiada żadnych dodatkowych funkcjonalności, gdyż służy tylko i wyłącznie
do kontroli procesu przez operatora.\\

W górnej części panelu, znajduje się pasek nawigacyjny. Po jego lewej stronie znajdują się 
przyciski do zmiany aktualnie wyświetlanego panelu. Aby przejść do panelu serii danych, należy
nacisnąć przycisk \textit{Serie danych}.\\

\subsection{Panel serii danych}
Panel serii danych składa się z dwóch sekcji. Po prawej stronie znajduje się panel służący do wyboru
urządzenia oraz przedziału czasowego. W celu wybrania przedziału należy wcisnąć pierwszy z dwóch
białych prostokątów. Spowoduje to pojawienie się kalendarza z możliwością wyboru godziny. Po wybraniu
początkowej daty i godziny należy wybrać drugi z prostokątów i zaznaczyć datę i godzinę końca przedziału.\\

Następnie należy wybrać, dane z których urządzeń chcemy serializować. Wyboru urządzenia dokonujemy 
poprzez zaznaczenie odpowiedniego przycisku. Istnieje możliwość wyboru kilku urządzeń równocześnie.
Wciśnięcie przycisku \textit{Pobierz}, spowoduje pobranie zserializowanych danych i ich 
prezentację na wykresie po lewej stronie panelu. \\

Po zakończeniu pracy z aplikacją, zalecanym jest wylogowanie się z systemu. Aby tego dokonać,
należy w prawej części paska nawigacyjnego, nacisnąć na swój login. Spowoduje to wylogowanie się 
z aplikacji i przekierowanie na panel logowania.

\section{Podsumowanie prac}
W specyfikacji wstępnej opisaliśmy następującą listę funkcjonalności:
\begin{itemize}
    \item Zbieranie danych od czujników
    \item Zapis pomiaru do bazy danych
    \item Wyświetlanie i odświeżanie aktualnych wartości zmiennych procesowych
    \item Pobranie danych pomiarowych z bazy i ich serializacja
    \item Generacja wykresów danych z zadaną dokładnością
    \item Możliwość łączenia wykresów
    \item Uwierzytelnianie
    \item Prezentacja informacji o urządzeniu
\end{itemize}

Wszystkie założone przez nas funkcjonalności zostały zrealizowane. Aplikacja działa
na systemie Linux (Ubuntu) i Windows. Dodatkowo, przygotowana została produkcjyna 
baza danych do przechowywania pomiarów, tak aby wszyscy użytkownicy mogli działać na tych 
samych danych.

\section{Opis rozwiązania}
\subsection{Architektura rozwiązania}
System składa się z czterech komponentów:
\begin{itemize}
\item Serwera panelu operatorskiego (frontend)
\item Serwera pośredniczącego (backend)
\item biblioteki kontrolerów 
\item Relacyjnej bazy danych
\end{itemize}
Panel operatorski został napisany w języku Javascript, przy wykorzystaniu
biblioteki React.js. Serwer pośredniczący został przygotowany w języku Python przy uży-
ciu pakietu Flask, z którego wywołujemy kontrolery transakcji obsługujące żądania
napisane w języku C++, połączone z serwerem przy wykorzystaniu biblioteki
Boost.Python. Do przechowywania danych wykorzystujemy bazę opartą o silnik PostgreSQL
lokalnie oraz w chmurze.

\subsection{Miary wielkościowe i jakościowe projektu}
Liczba linii kodu:
\begin{itemize}
    \item serwer frontendowy (Javascript): 1000 linii kodu
    \item serwer pośredniczący (Python): 100 linii kodu
    \item biblioteka kontrolerów (C++): 500 linii kodu
\end{itemize}

Procentowe pokrycie testami:
\begin{itemize}
    \item serwer frontendowy (Javascript): $20$ \%
    \item serwer pośredniczący (Python): $40$ \%
    \item biblioteka kontrolerów (C++): $90$ \%
\end{itemize}
Dodatkowo, przygotowaliśmy testy REST API przy pomocy programu Postman oraz testy
GUI przy pomocy programu Selenium. Obliczenie procentowego pokrycia tymi testami 
jest niemożliwe, jednak naszym zdaniem uzupełniają one testy modułów w języku Python
i Javascript.\\

Liczba osobogodzin spędzona nad projektem wyniosła około 80.
\section{Podsumowanie}
W trakcie pracy nad projektem, napotkaliśmy spory problem z instalacją biblioteki \texttt{libpqxx}
obsługującego komunikację z bazą danych na poziomie języka C++. Instalacja tej biblioteki w systemie Windows
okazała się zagadnieniem nietrywialnym, z powodu niejednoznacznej instrukcji instalacji. Nie chcieliśmy 
rezygnować z tej biblioteki, ponieważ przygotowaliśmy już spory kawałek kodu z jej wykorzystaniem. Można było ustrzec się
przed tym problemem, najpierw przygotowując równolegle oba środowiska a dopiero potem rozpocząć
proces pisania kodu. 

\end{document}